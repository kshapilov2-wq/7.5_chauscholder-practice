\documentclass[12pt]{article}

% ------------------ Пакеты ------------------
\usepackage{ucs}
\usepackage[utf8x]{inputenc}
\usepackage[T2A]{fontenc}
\usepackage[russian]{babel}
\usepackage{amsmath, amssymb}
\usepackage{array, booktabs, graphicx, multirow}
\usepackage{geometry}
\geometry{a4paper, margin=2.5cm}

% ------------------ Начало документа ------------------
\begin{document}

\section*{Оценка числа операций и сравнение методов}



\subsection*{QR-разложение: метод вращения и отражения Хаусхолдера}

\textbf{Отражения Хаусхолдера.}
Для плотной $n\times n$ матрицы построение только матрицы $R$ требует примерно
\[
\frac{4}{3}\,n^{3} + \mathcal{O}(n^{2})
\]
арифметических операций.
Если нужно явно построить матрицу $Q$, то добавляется ещё примерно
\[
\frac{2}{3}\,n^{3} + \mathcal{O}(n^{2}),
\]
так что полное построение $Q$ и~$R$ стоит
\[
2\,n^{3} + \mathcal{O}(n^{2}).
\]
Применение отражений к правой части $b$ требует $\mathcal{O}(n^{2})$ операций.

\medskip
\textbf{Метод вращения.}
Для плотного случая зануление каждого поддиагонального элемента отдельным вращением
даёт трудоёмкость порядка
\[
2\,n^{3} + \mathcal{O}(n^{2})
\]
при явном накоплении матрицы~$Q$.
Применение тех же вращений к вектору~$b$ также требует $\mathcal{O}(n^{2})$ операций.

\medskip
\noindent
\textbf{Итог по QR.}
В плотном случае метод Хаусхолдера быстрее по ведущему члену,
если требуется только матрица~$R$ ($\tfrac{4}{3}n^{3}$ против~$\approx 2n^{3}$).
При построении обоих множителей $Q$ и~$R$ оба метода дают около~$2n^{3}$ операций,
но на практике Хаусхолдер предпочтительнее из-за лучшей векторизации и меньшей константы.

\subsection*{Сравнение с ранее рассмотренными методами факторизации}

Для сопоставления приведём ведущие члены по~$n$ (для плотных $n\times n$ матриц,
без учёта решения треугольных систем):

\begin{center}
\renewcommand{\arraystretch}{1.2}
\begin{tabular}{|l|c|}
\hline
\textbf{Метод} & \textbf{Ведущий член числа операций} \\
\hline
LU (метод Гаусса с частичным выбором главного элемента)
& $\displaystyle \frac{2}{3}\,n^{3} + \mathcal{O}(n^{2})$ \\
\hline
Холецкий ($A$ симм.\ положительно определена)
& $\displaystyle \frac{1}{3}\,n^{3} + \mathcal{O}(n^{2})$ \\
\hline
QR (Хаусхолдер), только $R$
& $\displaystyle \frac{4}{3}\,n^{3} + \mathcal{O}(n^{2})$ \\
\hline
QR (Хаусхолдер), явные $Q$ и $R$
& $\displaystyle 2\,n^{3} + \mathcal{O}(n^{2})$ \\
\hline
QR (Вращение), явные $Q$ и $R$
& $\displaystyle \approx 2\,n^{3} + \mathcal{O}(n^{2})$ \\
\hline
\end{tabular}
\end{center}

\subsection*{Стоимость решения $Ax=b$ после факторизации}

\begin{itemize}
  \item \textbf{LU / Холецкий:} два треугольных прогона $Ly=b$, $Ux=y$
  (или $R^{T}R\,x=b$) требуют порядка $\mathcal{O}(n^{2})$ операций для каждой правой части.
  \item \textbf{QR:} вычисление $y=Q^{T}b$ (применение отражений или вращений)
  и обратный ход $Rx=y$ также в сумме дают $\mathcal{O}(n^{2})$.
\end{itemize}

\subsection*{Выводы}

\begin{itemize}
  \item По трудоёмкости (плотный случай):\\[2pt]
        Холецкий $<$ LU $<$ QR(Хаусхолдер, только $R$) $<$ QR(явные $Q,R$).
  \item Для $A$ симметричной положительно определённой матрицы оптимален метод Холецкого
        (наименее затратный и устойчивый).
  \item Для общей матрицы $A$ QR-метод устойчивее LU, но требует большего числа операций.
  \item Внутри QR-методов: Хаусхолдер эффективнее для плотных матриц,
        Вращение — для разреженных и при локальных обновлениях.
\end{itemize}

\end{document}